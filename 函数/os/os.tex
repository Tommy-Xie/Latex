\documentclass[11pt]{article}
\usepackage{ctex}

\usepackage[left=1.25in,right=1.25in,top=1in,bottom=1in]{geometry}

\input{mp-style2.tex}

\usepackage{fontspec}  

\begin{document}
\section{os 操作系统接口模块}
\subsection{os.makedirs}
os.makedirs(name, mode=O0777, exist\_ok=False)
递归目录创建函数。会自动创建到达最后一级目录所需要的中间目录。
如果 exist\_ok 为 False (默认值),则如果目标目录已存在将引发 FileExistsError
\begin{Python}{新建文件夹}
import os

def create_directory_if_not_exists(path):
"""
Creates 'path' if it does not exist
"""
os.makedirs(path) #创建一个文件夹

path = "D:\python_code\RPMNet-master\RPMNet-master\problem"
create_directory_if_not_exists(path)
\end{Python}
\subsection{os.environ}
一个表示字符串环境的mapping对象。例如,environ['HOME'] 是你的主目录(在某些平台上)的路径名
\begin{Python}{设置多GPU运行}
import os
import torch
import argparse

parser = argparse.ArgumentParser()
parser.add_argument('--gpu', default=0, type=int, metavar='DEVICE',
help='GPU to use, ignored if no GPU is present. Set to negative to use cpu')
_args = parser.parse_args()
if _args.gpu >= 0:
	os.environ['CUDA_VISIBLE_DEVICES'] = str(_args.gpu)  #当你的电脑有两块以上GPU时,上面这两句代码才起作用!
	_device = torch.device('cuda:0') if torch.cuda.is_available() else torch.device('cpu')
else:
	_device = torch.device('cpu')

# def set_gpus(gpu_index):
#     if type(gpu_index) == list:
#         gpu_index = ','.join(str(_) for _ in gpu_index)
#     if type(gpu_index) ==int:
#         gpu_index = str(gpu_index)
#     os.environ["CUDA_VISIBLE_DEVICES"] = gpu_index
\end{Python}
\begin{Python}{os.environ.keys}
windows:
os.environ[‘HOMEPATH’]:当前用户主目录。

os.environ[‘TEMP‘]:临时目录路径。

os.environ[PATHEXT’]:可执行文件。

os.environ[‘SYSTEMROOT‘]:系统主目录。

os.environ[‘LOGONSERVER’]:机器名。

os.environ[‘PROMPT’]:设置提示符。

linux:
os.environ[‘USER‘]:当前使用用户。

os.environ[‘LC_COLLATE’]:路径扩展的结果排序时的字母顺序。

os.environ[‘SHELL’]:使用shell的类型。

os.environ[‘LAN’]:使用的语言。

os.environ[‘SSH_AUTH_SOCK‘]:ssh的执行路径。
\end{Python}
\section{os.path 常用路径操作}
\subsection{join函数}
合理地拼接一个或多个路径部分。如果参数中某个部分是绝对路径,则绝对路径前的路径都将被丢弃,并从绝对路径部分开始连接。在Windows上,遇到绝对路径部分时,不会重置盘符。如果某部分路径包含盘符,则会丢弃所有先前的部分,并重置盘符。
\begin{Python}{拼接路径}
import os

problem_path = 'D:\python_code\Tensorflow'
log_path = os.path.join(problem_path, "problem.json")
print(log_path)
#OUTPUT:
#       D:\python_code\Tensorflow\problem.json
log_path = os.path.join('D:\python_code\Tensorflow', "problem.json")
print(log_path)
#OUTPUT:
#       D:\python_code\Tensorflow\problem.json
log_path = os.path.join('D:\python_code\pythonProject', "D:\python_code\Tensorflow")
print(log_path)
#OUTPUT:
#       D:\python_code\Tensorflow
\end{Python}

\end{document}