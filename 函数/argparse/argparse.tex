\documentclass[11pt]{article}
\usepackage{ctex}

\usepackage[left=1.25in,right=1.25in,top=1in,bottom=1in]{geometry}

\input{mp-style2.tex}

\usepackage{fontspec}  % for Consolas

\begin{document}
	运行的文件名为code.py
	注意positional arguments 和optional arguments中的内容
	
	\begin{Python}{两种输出方式}
		parser.print_help()
		args = parser.parse_args()
	\end{Python}

\section{Class : argparse.ArgumentParser }
	\begin{Python}{argparse.ArgumentParser参数}
		class argparse.ArgumentParser
		   (prog=None,
		   usage=None,
		   description=None,
		   epilog=None,
		   parents=[],
		   formatter_class=argparse.HelpFormatter,
		   prefix_chars='-',
		   fromfile_prefix_chars=None,
		   argument_default=None,
		   conflict_handler='error',
		   add_help=True,
		   allow_abbrev=True)
	\end{Python}
	
	创建一个新的 ArgumentParser 对象。所有的参数都应当作为关键字参数传入。   
	
\subsection{无指定参数}
	\begin{Python}{无参数输入}
import argparse
parser = argparse.ArgumentParser()
args = parser.parse_args()
		
#  注意 parser.parse_args 所在位置,输入格式python 文件名 -h
	 
#Output:  
#		 usage: code.py [-h]

#        optional arguments:
#          -h, --help  show this help message and exit
     \end{Python}
 
\subsection{prog参数} 	
 	\begin{Python}{prog - 程序的名称(默认值:sys.argv[0])}
import argparse
parser = argparse.ArgumentParser(prog='myprogram')
args = parser.parse_args()
 		
#Output:   
#		usage: myprogram [-h],         显示了prog里的字符串
 		
#       optional arguments:
#         -h, --help  show this help message and exit
 	\end{Python} 
 
\subsection{usage参数} 	
 	\begin{Python}{usage - 描述程序用途的字符串(默认值:从添加到解析器的参数生成)}
import argparse
parser = argparse.ArgumentParser(prog='PROG', usage='\%(prog)s [option]')
args = parser.parse_args()
 		
#Output:   
#		 usage: PROG [options] 

#        optional arguments:
#          -h, --help  show this help message and exit
 	\end{Python}
 	在用法消息中可以使用 (prog)s 格式说明符来填入程序名称。通过 usage= [关键字参数]覆盖这一默认消息即-h
 
\subsection{description参数}	 
 	\begin{Python}{description - 在参数帮助文档之前显示的文本(默认值:无)}
 		
import argparse
parser = argparse.ArgumentParser(description='A foo that bars')
args = parser.parse_args()
 		
#Output:
#		usage: code.py [-h]

#       A foo that bars

#       optional arguments:
#         -h, --help  show this help message and exit
 		
 	\end{Python}
 	在帮助消息中,这个描述会显示在usage:和optional arguments:之间,新加上的。
 	
\subsection{epilog参数}
 	\begin{Python}{epilog - 在参数帮助文档之后显示的文本(默认值:无)}
import argparse
parser = argparse.ArgumentParser(epilog="And that's how you'd foo a bar")
parser.print_help()
 		
#Output:
#		usage: code.py [-h]

#       optional arguments:
#         -h, --help  show this help message and exit

#       And that's how you'd foo a bar
 	\end{Python}
 
\subsection{parents参数}
 	\begin{Python}{parents - 一个 ArgumentParser 对象的列表,它们的参数也应包含在内}
import argparse
parent_parser = argparse.ArgumentParser(add_help=False)
parent_parser.add_argument('--parent', type=int)
 		
foo_parser = argparse.ArgumentParser(parents=[parent_parser])
foo_parser.add_argument('foo')
arg = foo_parser.parse_args(['--parent', '2', 'XXX'])
print(arg)
 	
#Output: 
#		Namespace(foo='XXX', parent=2)
 		
import argparse
parent_parser = argparse.ArgumentParser(add_help=False)
parent_parser.add_argument('--parent')
 		
bar_parser = argparse.ArgumentParser(parents=[parent_parser])
bar_parser.add_argument('--bar')
arg = bar_parser.parse_args(['--bar', 'YYY'])
print(arg)
 		
#Output:
#		Namespace(bar='YYY', parent=None)

注意大多数父解析器会指定 add_help=False.否则ArgumentParse将会看到两个 -h/--help 选项(一个在父参数中一个在子参数中)并且产生一个错误

import argparse
#父参数
p_parser = argparse.ArgumentParser(add_help = False)
p_parser.add_argument('-f')
#子参数并使用父参数
son = argparse.ArgumentParser(parents = [p_parser])
son.add_argument('-p')
sonargs = son.parse_args()
print(sonargs)
 		
# Input:
#		python code.py -p aa -f bb
# Output:
#		Namespace(f='bb', p='aa')
 	\end{Python}
 
\subsection{formatter\_class参数}
\subsubsection{formatter\_class = argparse.RawDescriptionHelpFormatter}
	formatter\_class=argparse.RawDescriptionHelpFormatter 表示 description 和 epilog 已经被正确的格式化了,不会在命令行中被改变输出格式:
 	\begin{Python}{argparse.RawDescriptionHelpFormatter}
import argparse
import textwrap
parser = argparse.ArgumentParser\
		(
		formatter_class=argparse.RawDescriptionHelpFormatter,
		description=textwrap.dedent
			('''
				Please do not mess up this text!
				--------------------------------
				I have indented it
				exactly the way
				I want it
			'''),
		epilog = '''
			likewise for this epilog whose whitespace will
			be cleaned up and whose words will be wrapped
			across a couple lines'''
		)
parser.print_help()
	
#Output:
#		usage: code.py [-h]

#		Please do not mess up this text!
#		--------------------------------
#		I have indented it
#		exactly the way
#		I want it

#		optional arguments:
#		-h, --help  show this help message and exit
#		likewise for this epilog whose whitespace will
#		be cleaned up and whose words will be wrapped
#		across a couple lines
	\end{Python}
\subsubsection{formatter\_class = argparse.RawTextHelpFormatter}
	formatter\_class=argparse.RawTextHelpFormatter 保留所有种类文字空格。但是多个新行被替换为一行。需保留多个空行,请在换行符之间添加空格
	\begin{Python}{argarse.RawTextHeilFormatter}
import argparse
parser = argparse.ArgumentParser(formatter_class=argparse.RawTextHelpFormatter,
description='hello\n \n \n world!! \n\n\n\n I love Python')
parser.print_help()
		
#Output:
#		usage: code.py [-h]

#		hello


#		world!!

#		I love Python

#		optional arguments:
#			-h, --help  show this help message and exit

	\end{Python}
	
\subsubsection{formatter\_class = argparse.ArgumentDefaultsHelpFormatter}
		formatter\_class=ArgumentDefaultsHelpFormatter 自动将有关默认值的信息添加到每个参数帮助消息中
	\begin{Python}{argarse.ArgumentDefaultsHelpFormatter}
import argparse
parser = argparse.ArgumentParser(formatter_class=argparse.ArgumentDefaultsHelpFormatter)
parser.add_argument('--foo', type=int, default= 42, help='FOO')
parser.add_argument('bar', nargs='*', default=[1, 2, 3], help='BAR!')
parser.print_help()
		
#Output:
#		usage: code.py [-h] [--foo FOO] [bar [bar ...]]
		
#		positional arguments:
#			bar         BAR! (default: [1, 2, 3])
		
#		optional arguments:
#		  -h, --help  show this help message and exit
#		  --foo FOO   FOO (default: 42)
		
	\end{Python}
\subsubsection{formatter\_class = argparse.MetavarTypeHelpFormatter}
	formatter\_class=MetavarTypeHelpFormatter  它的值在每一个参数中使用 type 的参数名当作它的显示名(而不是使用通常的格式 dest)
	\begin{Python}{argparse.MetavarTypeHelpFormatter}
import argparse
parser = argparse.ArgumentParser(formatter_class=argparse.MetavarTypeHelpFormatter)
parser.add_argument('foo', type=int)
parser.add_argument('--bar', type=float)
parser.print_help()

#Output:
#		usage: code.py [-h] [--bar float] int
		
#		positional arguments:
#		int
		
#		optional arguments:
#		  -h, --help   show this help message and exit
#		  --bar float
		
	\end{Python}
\subsection{prefix\_chars参数}
  prefix\_chars= 参数默认使用 '-'. 支持一系列字符,但是不包括 - ,这样会产生不被允许的 -f/--foo 选项
 
 	\begin{Python}{prefix\_chars- 可选参数的前缀字符集合(默认值: '-')}
import argparse
parser = argparse.ArgumentParser(prefix_chars='+')
parser.add_argument('++foo', type=int)
parser.print_help()

# Output:
#		 usage: code.py [+h] [++foo FOO]

#		 optional arguments:
#          +h, ++help  show this help message and exit
#          ++foo FOO

import argparse
parser = argparse.ArgumentParser(prefix_chars='+')
parser.add_argument('--foo', type=int)
parser.print_help()

#Output:
#		usage: code.py [+h] --foo

#		positional arguments:
#  		  --foo

#       optional arguments:
#         +h, ++help  show this help message and exit

 	\end{Python}
\subsection{fromfile\_prefix\_chars参数}
 fromfile\_prefix\_chars - 当需要从文件中读取其他参数时,用于标识文件名的前缀字符集合(默认值: None)
 \begin{Python}{fromfile\_prefix\_chars-用于标识文件名的前缀字符集合(默认值: None)}
import argparse

with open('args.txt', 'w') as fp:
fp.write('-f\nbar')
parser = argparse.ArgumentParser(fromfile_prefix_chars='%')
parser.add_argument('-f')
arg = parser.parse_args(['-f', 'foo', '%args.txt'])
print(arg)

#Output:
#		Namespace(f='bar')

 \end{Python}
 
 当参数过多时,可以将参数放到文件中读取,
 例子中parser.parse\_args(['-f', 'foo', '@args.txt'])
 解析时会从文件args.txt读取,相当于 ['-f', 'foo', '-f', 'bar']。
 
\subsection{argument\_default参数}
 	一般情况下,参数默认会通过设置一个默认到 add\_argument() 或者调用带一组指定键值对的ArgumentParser.set\_defaults() 方法。
 	
 	\begin{Python}{提供argument\_default为argperse.SUPPRESS}
 		
import argparse

parser = argparse.ArgumentParser(argument_default=argparse.SUPPRESS)
parser.add_argument('--foo')
parser.add_argument('bar')
arg =parser.parse_args(['--foo', '1', 'BAR'])
print(arg)

#Output:
#		Namespace(bar='BAR', foo='1')

 	\end{Python}
 全局禁止在parse\_args中创建属性
\subsection{conflict\_handler参数}
 默认情况下, ArgumentParser 对象会产生一个异常如果去创建一个正在使用的选项字符串参数,有些时候(例如:使用 parents),重写旧的有相同选项字符串的参数会更有用。为了产生这种行为,conflict\_handler='resolve' :
 	\begin{Python}{conflict\_handler为'resolve'创建正在使用的参数}
import argparse

parser = argparse.ArgumentParser(prog='PROG', conflict_handler='resolve')
parser.add_argument('-f', '--foo', help='old foo help')
parser.add_argument('--foo', help='new foo help')
parser.print_help()

#Output:
#		usage: PROG [-h] [-f FOO] [--foo FOO]

#		optional arguments:
#		-h, --help  show this help message and exit
#		-f FOO      old foo help
#		--foo FOO   new foo help

 	\end{Python}
\subsection{add\_help参数}
 	隐藏了 -h, --help  show this help message and exit的信息
 	
	\begin{Python}{add\_help参数默认为Ture}
		
import argparse

parser = argparse.ArgumentParser(add_help=False)    
parser.print_help() 

#Output:
#		usage: code.py
	\end{Python}
\subsection{allow\_abbrev参数}
 allow\_abbrev 为 False 来关闭向 ArgumentParser 的 parse\_args() 方法传入一个参数列表。无法使用parser.parse\_args。
 	\begin{Python}{allow\_abbrev为False关闭参数传递}
import argparse

parser = argparse.ArgumentParser(allow_abbrev=False)
parser.add_argument('--foobar')
parser.add_argument('--foonley')
parser.parse_args(['--foon'])

#Output:
#		usage: code.py [-h] [--foobar FOOBAR] [--foonley FOONLEY]
#		code.py: error: unrecognized arguments: --foon
	
 	\end{Python}
 
\section{Class : argparse.ArgumentParser.add\_argument}
 	\begin{Python}{argparse.ArgumentParser.add\_argument参数}
 		ArgumentParser.add_argument
 		(
 		name or flags...
 		[action]
 		[nargs]
 		[const]
 		[default]
 		[type]
 		[choices]
 		[required]
 		[help]
 		[metavar]
 		[dest]
 		)
 	\end{Python}
 
\subsection{name or flags参数}
 	add\_argument() 方法必须知道它是否是一个选项或是一个位置参数,例如一组文件名
 	\begin{Python}
import argparse

parser = argparse.ArgumentParser(prog='PROG')
parser.add_argument('-f', '--foo')
parser.add_argument('bar')
parser.parse_args(['BAR'])

#Output:
#		Namespace(bar='BAR', foo=None)

import argparse
parser = argparse.ArgumentParser(prog='PROG')
parser.add_argument('-f', '--foo')
parser.add_argument('bar')
parser.parse_args(['BAR', '--foo', 'FOO'])

#Output:
#		Namespace(bar='BAR', foo='FOO')

import argparse
parser = argparse.ArgumentParser(prog='PROG')
parser.add_argument('-f', '--foo')
parser.add_argument('bar')
parser.parse_args(['--foo', 'FOO'])

#Output:
#		usage: PROG [-h] [-f FOO] bar
#		PROG: error: the following arguments are required: bar
 	\end{Python}
\subsection{action参数}
 	
 	大多数动作只是简单的向 parse\_args() 返回的对象上添加属性
 	
\subsubsection{action存储默认参数}
 	
 	\begin{Python}{不指定action的值}
 	
import argparse

parser = argparse.ArgumentParser()
parser.add_argument('--foo')
arg = parser.parse_args('--foo 1'.split())
print(arg)

#Output:
#		Namespace(foo='1')		
 	\end{Python}
\subsubsection{'store\_const'}
 	'store\_const' 动作通常用来存储被const命名参数指定的元素。
 	\begin{Python}{'store\_const'为数字}
import argparse

parser = argparse.ArgumentParser()
parser.add_argument('--foo', action='store_const', const=42)
arg = parser.parse_args(['--foo'])
print(arg)

#Output:
#		Namespace(foo=42)
 	\end{Python}
\subsubsection{'store\_true'和'store\_false'}
'store\_true'和'store\_false'分别用作存储True和False值的特殊用例
\begin{Python}{store\_true默认值True;store\_false默认值False}
import argparse

parser = argparse.ArgumentParser()
parser.add_argument('--foo', action='store_true')
parser.add_argument('--bar', action='store_false')
parser.add_argument('--baz', action='store_false')
# arg = parser.parse_args('--foo'.split())
# print(arg)

#Output:
#       Namespace(bar=True, baz=True, foo=True)

arg = parser.parse_args('--foo --bar'.split())
print(arg)

#Output:
#       Namespace(bar=False, baz=True, foo=True)
\end{Python}
\subsubsection{'append'}
存储一个列表,并且将每个参数值追加到列表中。
\begin{Python}{存储列表}
import argparse

parser = argparse.ArgumentParser()
parser.add_argument('--foo', action='append')
arg = parser.parse_args('--foo 1 --foo auifa'.split())
print(arg)

#Output:
#       parser.add_argument('--foo', action='append')	
\end{Python}
\subsubsection{'append\_const'}
存储一个列表,并将conse命名参数指定的值追加到列表中,默认为None。该动作一般在多个参数需要在同一列表中存储常数时会有用
\begin{Python}{参见下面的dest用法}
import argparse

parser = argparse.ArgumentParser()
parser.add_argument('--str', dest='types',action='append_const', const=str )
parser.add_argument('--int1', action='append_const', const=int)
parser.add_argument('--int', dest='jkhd',action='append_const', const=int)
arg = parser.parse_args('--str --int --int1'.split())
print(arg)

#Output:
#       Namespace(int1=[<class 'int'>], jkhd=[<class 'int'>], types=[<class 'str'>])
\end{Python}
\subsubsection{'count'}
计算一个关键字参数出现的数目或次数.注意次数能够自己指定,也能是默认,默认为0.
\begin{Python}
import argparse

parser = argparse.ArgumentParser()
parser.add_argument('--count', '-v',action='count', default=1)
arg = parser.parse_args(['-vvv'])
print(arg)

#Output:
#       Namespace(count=4)
\end{Python}
\subsubsection{'help'}
打印所有当前解析器中的选项和参数的完整帮助信息,然后退出。默认情况下,一个 help 动作会被自动加入解析器。
\begin{Python}{加入帮助信息}
import argparse

parser = argparse.ArgumentParser()
parser.add_argument('-count', '--hjsdv',action='help')
arg = parser.parse_args()
print(arg)

#Output:
#       usage: Argparse.py [-h] [--count]
#
#       optional arguments:
#           -h, --help       show this help message and exit
#           -count, --hjsdv
\end{Python}
\subsubsection{'version'}
能够打印版本信息并在调用后退出
\begin{Python}{打印版本信息}
import argparse

parser = argparse.ArgumentParser(prog='PROG')
parser.add_argument('--version',action='version',version='%(prog)s 2.0')
arg = parser.parse_args(['--version'])
print(arg)

#Output:
#       PROG 2.0
\end{Python}
\subsubsection{自定义函数}
还可以通过传递 Action 子类或实现相同接口的其他对象来指定任意操作。
\begin{Python}{自定义}
import argparse

class FooAction(argparse.Action):
def __init__(self, option_strings, dest, nargs=None, **kwargs):
if nargs is not None:
raise ValueError("nargs not allowed")
super(FooAction,self).__init__(option_strings, dest,**kwargs)
def __call__(self, parser, namespace, values, option_string=None):
print(' %r %r %r' % (namespace, values, option_string))
setattr(namespace, self.dest, values)

parser = argparse.ArgumentParser()
parser.add_argument('--foo', action=FooAction)
parser.add_argument('bar', action=FooAction)
args = parser.parse_args('1, --foo 2'.split())
#OUTPUT:
#        Namespace(bar=None, foo=None) '1,' None
#        Namespace(bar='1,', foo=None) '2' '--foo'
print(args)
#OUTPUT:
#        Namespace(bar='1,', foo='2')
\end{Python}
\subsection{nargs参数}
nargs命名参数关联不同数目的命令行参数到单一动作。
\subsubsection{N(一个整数)}
命令行中的N个参数会被聚集到一个列表中。
\begin{Python}{nargs=N}
import argparse

parser = argparse.ArgumentParser()
parser.add_argument('--foo', nargs=2)
parser.add_argument('bar', nargs=2)
arg = parser.parse_args('c d --foo a b'.split())
print(arg)

#OUTPUT:
#       Namespace(bar=['c', 'd'], foo=['a', 'b'])
\end{Python}
\subsubsection{?}
如果可能的话,会从命令行中消耗一个参数,并产生一个单一项。如果当前没有命令行参数,则会产生 default 值。
\begin{Python}{允许可选的输入或输出文件}
import argparse
import sys

parser = argparse.ArgumentParser()
parser.add_argument('infile', nargs='?', type=argparse.FileType('r'), default=sys.stdin)
parser.add_argument('outfile', nargs='?', type=argparse.FileType('w'), default=sys.stdout)
arg = parser.parse_args(['input.txt', 'output.txt'])
print(arg)

#OUTPUT:
#       Namespace(infile=<_io.TextIOWrapper name='input.txt' mode='r' encoding='cp936'>, outfile=
# <_io.TextIOWrapper name='output.txt' mode='w' encoding='cp936'>)

arg = parser.parse_args([])
print(arg)

#OUTPUT:
#       Namespace(infile=<_io.TextIOWrapper name='<stdin>' mode='r' encoding='utf-8'>, outfile=
# <_io.TextIOWrapper name='<stdout>' mode='w' encoding='utf-8'>)
\end{Python}
\subsubsection{*}
所有当前命令行参数被聚集到一个人列表中。
\begin{Python}{多个选项进行分配}
import argparse

parser = argparse.ArgumentParser()
parser.add_argument('--foo', nargs='*')
parser.add_argument('bar', nargs='*')
parser.add_argument('--baz', nargs='*')
args = parser.parse_args('a b --foo x y --baz 1 2'.split())
print(args)

#OUTPUT:
#       Namespace(bar=['a', 'b'], baz=['1', '2'], foo=['x', 'y'])
\end{Python}
\subsubsection{+}
所有当前命令行参数被聚集到一个列表中。
\begin{Python}{没有命令行参数时报错}
import argparse

parser = argparse.ArgumentParser()
parser.add_argument('foo', nargs='+')
args = parser.parse_args(['a', 'b'])
print(args)

#OUTPUT:
#       Namespace(foo=['a', 'b'])

args = parser.parse_args([])
print(args)

#OUTPUT:
#       usage: Argparse.py [-h] foo [foo ...]
#       Argparse.py: error: the following arguments are required: foo
\end{Python}
\subsubsection{argarse.REMAINDER}
剩余的命令行参数被聚集到一个列表中。
\begin{Python}{命令行功能的传递}
import argparse

parser = argparse.ArgumentParser(prog='PROG')
parser.add_argument('--foo')
parser.add_argument('command')
parser.add_argument('args', nargs=argparse.REMAINDER)
print(parser.parse_args('--foo B cmd --arg1 XX ZZ'.split()))

#OUTPUT:
#       Namespace(args=['--arg1', 'XX', 'ZZ'], command='cmd', foo='B')
\end{Python}
\subsection{const参数}
用于保存不从命令行中读取但被需求的常数值。

1.当 add\_argument() 通过 action='store\_const' 或 action='append\_const 调用时。这些动作将 const 值添加到 parse\_args() 返回的对象的属性中

2.当 add\_argument() 通过选项(例如 -f 或 --foo)调用并且 nargs='?' 时。这会创建一个可以跟随零个或一个命令行参数的选项。当解析命令行时,如果选项后没有参数,则将用 const 代替。
\begin{Python}{const提供默认值}
import argparse

parser = argparse.ArgumentParser()
parser.add_argument('--foo', nargs='?', const='c', default='d')
a = parser.parse_args(['--foo', 'YY'])
b = parser.parse_args(['--foo'])
c = parser.parse_args([])
print(a, b, c)

#OUTPUT:
#       Namespace(foo='YY') Namespace(foo='c') Namespace(foo='d')
\end{Python}
\subsection{default参数}
所有选项和一些位置参数可能在命令行中被忽略,用于指定在命令行参数未出现时应当使用的值。
\begin{Python}{常见应用1}
import argparse

parser = argparse.ArgumentParser()
parser.add_argument('--foo', default=42)
a = parser.parse_args(['--foo', '2'])
b = parser.parse_args([])
print(a, b)

#OUTPUT:
#       Namespace(foo='2') Namespace(foo=42)
\end{Python}
\begin{Python}{常见应用2}
import argparse

parser = argparse.ArgumentParser()
parser.add_argument('--length', default='10', type=int)
parser.add_argument('--width', default=10.535, type=int)
b = parser.parse_args([])
print(b)

#OUTPUT:
#       Namespace(length=10, width=10.535)
\end{Python}
nargs可以为?或*,注意分别。
\begin{Python}{常见运用3}
import argparse

parser = argparse.ArgumentParser()
parser.add_argument('length', nargs='*', default=42)     #Namespace(length=['a']) Namespace(length=42)  
parser.add_argument('width', nargs='?', default=20)     #Namespace(width='a') Namespace(width=20)
a = parser.parse_args(['a'])
b = parser.parse_args([])
print(a, b)

#OUTPUT:
#       Namespace(length=['a'], width=20) Namespace(length=42, width=20)
\end{Python}
\begin{Python}{常见运用4}
import argparse

parser = argparse.ArgumentParser()
parser.add_argument('--foo', default=argparse.SUPPRESS)
a = parser.parse_args([])
b = parser.parse_args(['--foo', '1'])
print(a,b)

#OUTPUT:
#       Namespace() Namespace(foo='1')
\end{Python}
\subsection{type参数}
type关键词参数允许任何的类型检查和类型转换。一般的内建类型和函数可以直接被type参数使用
\begin{Python}{规定类型}
import argparse

parser = argparse.ArgumentParser()
parser.add_argument('foo', type=int)
parser.add_argument('bar', type=open)
a = parser.parse_args('2 temp.txt'.split())
print(a)

#OUTPUT:
#       Namespace(bar=<_io.TextIOWrapper name='temp.txt' mode='r' encoding='cp936'>, foo=2)
\end{Python}
\begin{Python}{调用定义对象}
import argparse
import math

def perfect_square(string):
value = int(string)
sqrt = math.sqrt(value)
if sqrt != int(sqrt):
msg = "%r is not a perfect square" % string
raise argparse.ArgumentTypeError(msg)
return value

parser = argparse.ArgumentParser(prog='PROG')
parser.add_argument('foo', type=perfect_square)
a = parser.parse_args(['9'])
b = parser.parse_args(['7'])
print(a,b)

#OUTPUT:
#       Namespace(foo=9)
#       usage: PROG [-h] foo
#       PROG: error: argument foo: '7' is not a perfect square
\end{Python}
\subsection{choices参数}
某些命令行参数应当从一组受限值中选择。 这可通过将一个容器对象作为 choices 关键字参数传给 add\_argument() 来处理。
\begin{Python}{只能从中选择}
import argparse

parser = argparse.ArgumentParser(prog='doors.py')
parser.add_argument('door', type=int, choices=range(1, 4))
print(parser.parse_args(['3']))
# Namespace(door=3)
parser.parse_args(['4'])
#OUTPUT:
#       usage: doors.py [-h] {1,2,3}
#       doors.py: error: argument door: invalid choice: 4 (choose from 1, 2, 3)
\end{Python}
\subsection{required参数}
通常,argparse 模块会认为 -f 和 --bar 等旗标是指明可选的参数,它们总是可以在命令行中被忽略。 要让一个选项成为必需的,将required=True作为关键字参数传给 add\_argument()
\begin{Python}{一般不使用,用户预期是可选的参数}
import argparse

parser = argparse.ArgumentParser()
parser.add_argument('--foo', required=True)
a = parser.parse_args(['--foo', 'BAR'])
print(a)
#Namespace(foo='BAR')
b = parser.parse_args([])
print(b)
#OUTPUT:
#       usage: Argparse.py [-h] --foo FOO
#       Argparse.py: error: the following arguments are required: --foo
\end{Python}
\subsection{help参数}
help 值是一个包含参数简短描述的字符串。 当用户请求帮助时(一般是通过在命令行中使用-h或 --help的方式),这些help描述将随每个参数一同显示:

另外如果你希望在帮助字符串中显示 \% 字面值,你必须将其转义为 \%\%
\begin{Python}{显示简短描述}
import argparse

parser = argparse.ArgumentParser(prog='frobble')
parser.add_argument('--foo', action='store_true',
help='foo the bars before frobbling')
parser.add_argument('bar', nargs='?', type=int, default=42,
help='the bar tp %(prog)s (default: %(default)s)')
parser.parse_args(['-h'])

#OUTPUT:
#       usage: frobble [-h] [--foo] [bar]
# 
#       positional arguments:
#           bar         the bar tp frobble (default: 42)
# 
#       optional arguments:
#           -h, --help  show this help message and exit
#           --foo       foo the bars before frobbling
\end{Python}
\begin{Python}{静默特定选项的帮助}
import argparse

parser = argparse.ArgumentParser(prog='frobble')
parser.add_argument('--foo', help=argparse.SUPPRESS)
parser.add_argument('--bar', help='one of the bars to be frobbled')
parser.print_help()

#OUTPUT:
#       usage: frobble [-h] [--bar BAR]
#
#       optional arguments:
#           -h, --help  show this help message and exit
#           --bar BAR   one of the bars to be frobbled
\end{Python}
\subsection{metavar参数}
可以使用 metavar 来指定一个替代名称
\begin{Python}{名称指定}
import argparse

parser = argparse.ArgumentParser()
parser.add_argument('--foo', metavar='YYY')
parser.add_argument('bar', metavar='XXX')
parser.parse_args('X --foo Y'.split())

parser.print_help()

#OUTPUT:
#       usage: Argparse.py [-h] [--foo YYY] XXX
# 
#       positional arguments:
#           XXX
# 
#       optional arguments:
#           -h, --help  show this help message and exit
#           --foo YYY
\end{Python}
\subsection{dest参数}
对于可选参数动作,dest 的值通常取自选项字符串。
\begin{Python}{允许提供自定义属性名称}
import argparse

parser = argparse.ArgumentParser()
parser.add_argument('--foo', dest='bar')
a = parser.parse_args('--foo XXX'.split())
print(a)

#OUTPUT:
#       Namespace(bar='XXX')
\end{Python}
\section{Class : argparse.Action}
\section{其他}
\subsection{部分解析}
argparse.ArgumentParser.parse\_known\_args(args=None, namespace=None)。当仅获取到基本设置时,如果运行命令中传入了之后才会获取到的其他配置,不会报错;而是将多出来的部分保存起来,留到后面使用
\begin{Python}{argparse.parse\_known\_args}
import argparse
def basic_options():
parser = argparse.ArgumentParser()
parser.add_argument('--data_mode', type=str, default= 'unaligned', help='chooses how datasets are loaded')
parser.add_argument('--mode', type=str, default='test', help='test mode')
return parser

def data_options(parser):
parser.add_argument('--lr', type=str, default='0.0001', help='learning rate')
return parser

if __name__ == '__main__':
parser = basic_options()
opt, unparsed = parser.parse_known_args()
print(opt)
print(unparsed)
parser = data_options(parser)
opt = parser.parse_args()
print(opt)

#Input: python test_data.py --data_mode aligned --lr 0.0002

#Output:Namespace(data_mode='aligned', mode='test')
#       ['--lr', '0.0002']
#       Namespace(data_mode='aligned', lr='0.0002', mode='test')

\end{Python}
\end{document}