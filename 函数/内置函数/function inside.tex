\documentclass[11pt]{article}
\usepackage{ctex}

\usepackage[left=1.25in,right=1.25in,top=1in,bottom=1in]{geometry}

\input{mp-style2.tex}

\usepackage{fontspec}  % for Consolas

\begin{document}
\section{staticmethod}
staticmethod 返回函数的静态方法。

该方法不强制要求传递参数

\begin{Python}{@staticmethod方法}
class C():
def __init__(self):
self.f()  # 静态方法无需实例化   OUTPUT:runoob
@staticmethod
def f():
print('runoob');

cobj = C()
cobj.f()  # 也可以实例化后调用       OUTPUT:runoob
\end{Python}
\section{call}
对于可调用对象,实际上“名称()”可以理解为是“名称.\_\_call\_\_()”的简写。
\begin{Python}{}
class X(object):
	def __init__(self, a, b, range):
		self.a = a
		self.b = b
		self.range = range
	def __call__(self, a, b):
		self.a = a
		self.b = b
		print('__call__ with ({}, {})'.format(self.a, self.b))

xInstance = X(1, 2, 3)  #创建实例
xInstance(1,2)          #实例可以像函数那样执行,并传入a b值,修改对象的a b

#OUTPUT:
#		__call__ with (1, 2)
\end{Python}
\section{vars}
vars() 函数返回对象的 \_\_dic\_\_ 属性。
\_\_dict\_\_ 属性是包含对象的可变属性的字典。

注释:不带参数调用 vars() 函数将返回包含局部符号表的字典。
\begin{Python}
class Person:
name = "Bill"
age = 63
country = "USA"

x = vars(Person)

print(x)
#OUTPUT:
#       {'__module__': '__main__', 'name': 'Bill', 'age': 63, 'country': 'USA'}
\end{Python}
\section{future}
导入\_\_future\_\_支持的语言特征division(精确除法),当我们没有在程序中导入该特征时,"/"操作符执行的是截断除法(Truncating Division),当我们导入精确除法之后,"/"执行的是精确除法.此导入方法使用于Python 2.x。而Python 3.x,此方法是自带的。
\begin{Python}{精确的除法}
from __future__ import division

print(4/3)
#   1.3333333333333333
\end{Python}
\end{document}