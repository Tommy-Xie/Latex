\documentclass[openbib]{article}

\usepackage{color}
\usepackage{ctex}
\usepackage{mathtools}
\usepackage{amsmath}
\usepackage{graphicx,psfrag,epsfig}
% Copyright 20120 Liutao Tian, MIT License
% https://github.com/andy123t/code-latex-style/

\usepackage{listings,color}

% Matlab highlight color settings
%\definecolor{mBasic}{RGB}{248,248,242}       % default
\definecolor{mKeyword}{RGB}{0,0,255}          % bule
\definecolor{mString}{RGB}{160,32,240}        % purple
\definecolor{mComment}{RGB}{34,139,34}        % green
\definecolor{mBackground}{RGB}{245,245,245}   % lightgrey
\definecolor{mNumber}{RGB}{134,145,148}       % gray

\definecolor{Numberbg}{RGB}{237,240,241}     % lightgrey

% Python highlight color settings
%\definecolor{pBasic}{RGB}{248, 248, 242}     % default
\definecolor{pKeyword}{RGB}{228,0,128}        % magenta
\definecolor{pString}{RGB}{148,0,209}         % purple
\definecolor{pComment}{RGB}{117,113,94}       % gray
\definecolor{pIdentifier}{RGB}{166, 226, 46}  %
\definecolor{pBackground}{RGB}{245,245,245}   % lightgrey
\definecolor{pNumber}{RGB}{134,145,148}       % gray

\lstnewenvironment{Python}[1]{
	\lstset{language=python,               % choose the language of the code
		xleftmargin=30pt,
		xrightmargin=10pt,
		frame=l,
		framesep=15pt,%framerule=0pt,  % sets the frame style
		%frame=shadowbox,rulesepcolor=\color{red!20!green!20!blue!20},
		%basicstyle=\small\ttfamily,          % sets font style for the code
		basicstyle=\footnotesize\fontspec{Consolas},
		keywordstyle=\color{pKeyword},       % sets color for keywords
		stringstyle=\color{pString},         % sets color for strings
		commentstyle=\color{pComment},       % sets color for comments
		backgroundcolor=\color{pBackground}, % choose the background color
		title=#1,                            %\lstname show the filename of files
		emph={format_string,eff_ana_bf,permute,eff_ana_btr},
		emphstyle=\color{pIdentifier}
		showspaces=false,                    % show spaces adding particular underscores
		showstringspaces=false,              % underline spaces within strings
		showtabs=false,                      % show tabs within strings adding particular underscores
		tabsize=4,                           % sets default tabsize to 2 spaces
		captionpos=t,                        % sets the caption-position to bottom
		breaklines=true,                     % sets automatic line breaking
		framexleftmargin=5pt,
		fillcolor=\color{Numberbg},
		rulecolor=\color{Numberbg},
		numberstyle=\tiny\color{pNumber},
		numbersep=9pt,                      % how far the line-numbers are from the code
		numbers=left,                        % where to put the line-numbers
		stepnumber=1,                        % the step between two line-numbers.
}}{}

\lstnewenvironment{Python1}[1]{
\lstset{language=python,               % choose the language of the code
  xleftmargin=30pt,
  xrightmargin=10pt,
  frame=l,
  framesep=15pt,%framerule=0pt,  % sets the frame style
  %frame=shadowbox,rulesepcolor=\color{red!20!green!20!blue!20},
  %basicstyle=\small\ttfamily,          % sets font style for the code
  basicstyle=\footnotesize\fontspec{Consolas},
  keywordstyle=\color{pKeyword},       % sets color for keywords
  stringstyle=\color{pString},         % sets color for strings
  commentstyle=\color{pComment},       % sets color for comments
  backgroundcolor=\color{pBackground}, % choose the background color
  title=#1,                            %\lstname show the filename of files
  emph={format_string,eff_ana_bf,permute,eff_ana_btr},
  emphstyle=\color{pIdentifier}
  showspaces=false,                    % show spaces adding particular underscores
  showstringspaces=false,              % underline spaces within strings
  showtabs=false,                      % show tabs within strings adding particular underscores
  tabsize=4,                           % sets default tabsize to 2 spaces
  captionpos=t,                        % sets the caption-position to bottom
  breaklines=true,                     % sets automatic line breaking
  framexleftmargin=5pt,
  fillcolor=\color{Numberbg},
  rulecolor=\color{Numberbg},
  numberstyle=\tiny\color{pNumber},
  numbersep=9pt,                      % how far the line-numbers are from the code
  numbers=left,                        % where to put the line-numbers
  stepnumber=1,                        % the step between two line-numbers.
}}{}



\usepackage{fontspec}
\usepackage{bm}
\graphicspath{{figures/}}
\renewcommand{\contentsname}{\centerline{目录}}

\usepackage{multirow}
\begin{document}
	
	%	\caption{第一章  }
	\title{PyQt5}
	
	%	pip install pyinstaller
	%	pyinstaller.exe -F -w C:\Users\seed\PycharmProjects\untitled5\main.py
	
	\maketitle
	
	\newpage
	\tableofcontents
	\newpage
	\section{实例开启学习}
	\begin{Python}{实例}
import sys
from PyQt5.QtWidgets import QApplication,QWidget

if __name__ == '__main__':
	#创建类的实例
	app = QApplication(sys.argv)
	#创建一个窗口
	w = QWidget()
	#设置窗口的尺寸
	w.resize(400,200)
	#移动窗口
	w.move(300,300)
	
	#设置窗口的标题
	w.setWindowTitle('第一个基于PyQy5的桌面应用')
	#显示窗口
	w.show()
	
	#进入程序的主循环,并通过exit函数确保主循环安全结束
	sys.exit(app.exec_())
	\end{Python}
以上的案例为自己打印的,为了生成更为复杂的界面,要调用Anaconda中专门生成界面的软件QtDesigner命名为Start Designer,路径为$D:\backslash Anaconda3\backslash Library\backslash bin\backslash designer.exe$

文件菜单中的设置,工具——外部工具。添加新的外部工具,名称和描述自己加,程序指定上述路径,参数可写:工作目录指定要运行Qtdesigner的工作程序目录,

在项目对话框中单击右键,选定External tools中所加的外部工具,就能调用起Qtdesigner。

\subsection{转换文件格式}
将.ui格式文件(Qtdesigner保存格式)转换为.py格式。

调用终端在项目所在的目录下运行pyuic5 demo.ui -o demo.py还有更简便的是将上述的程序路径改为$D:\backslash Anaconda3\backslash Library\backslash bin\backslash pyuic5.bat$。参数改为$\$FileName\$ -o \$FileNameWithoutExtension\$.py$。工作目录改为$\$FileDir\$$。之后直接选中要转换的.ui文件之后右键调用External tools中的外部工具。

\subsection{布局}
\begin{center}
	1.水平布局
\end{center}
选中要布局的对话框,单机右键,选定布局(Horizontal Layout),选定水平布局。也能选定布局后,直接将元素拖进趋于内,注意要沿着边线放置,出现蓝线后放置。完成水平设置布局后保存为text.ui,转换文件格式生成text.py文件。
\begin{Python}{水平布局}
import sys
import text
from PyQt5.QtWidgets import QApplication,QMainWindow

if __name__ == '__main__':
	#创建类的实例
	app = QApplication(sys.argv)
	#创建一个窗口
	mainWindow = QMainWindow()
	#向主窗口上添加控件
	ui = text.Ui_MainWindow()
	ui.setupUi(mainWindow)
	mainWindow.show()
	
	#进入程序的主循环,并通过exit函数确保主循环安全结束
	sys.exit(app.exec_())
\end{Python}
\begin{center}
	2.垂直布局
\end{center}
选中要布局的对话框,单机右键,选定布局,选定垂直布局(Vertical Layout)。也能选定布局后,直接将元素拖进趋于内,注意要沿着边线放置,出现蓝线后放置。完成水平设置布局后保存为text.ui,转换文件格式生成text.py文件。
\begin{Python}{垂直布局}
import sys
import text
from PyQt5.QtWidgets import QApplication,QMainWindow

if __name__ == '__main__':
	#创建类的实例
	app = QApplication(sys.argv)
	#创建一个窗口
	mainWindow = QMainWindow()
	#向主窗口上添加控件
	ui = text.Ui_MainWindow()
	ui.setupUi(mainWindow)
	mainWindow.show()
	
	#进入程序的主循环,并通过exit函数确保主循环安全结束
	sys.exit(app.exec_())
\end{Python}
\begin{center}
	3.网格布局
\end{center}
选中要布局的对话框,单机右键,选定布局,选定网格布局(Grid Layout)。也能选定布局后,直接将元素拖进趋于内,注意要沿着边线放置,出现蓝线后放置。完成水平设置布局后保存为text.ui,转换文件格式生成text.py文件。
\begin{Python}{网格布局}
import sys
import text
from PyQt5.QtWidgets import QApplication,QMainWindow

if __name__ == '__main__':
	#创建类的实例
	app = QApplication(sys.argv)
	#创建一个窗口
	mainWindow = QMainWindow()
	#向主窗口上添加控件
	ui = text.Ui_MainWindow()
	ui.setupUi(mainWindow)
	mainWindow.show()
	
	#进入程序的主循环,并通过exit函数确保主循环安全结束
	sys.exit(app.exec_())
\end{Python}
\begin{center}
	4.表单布局
\end{center}
选中要布局的对话框,单机右键,选定布局,选定表单布局(Form Layout)。也能选定布局后,直接将元素拖进趋于内,注意要沿着边线放置,出现蓝线后放置。完成水平设置布局后保存为text.ui,转换文件格式生成text.py文件。
\begin{Python}{表单布局}
import sys
import text
from PyQt5.QtWidgets import QApplication,QMainWindow

if __name__ == '__main__':
	#创建类的实例
	app = QApplication(sys.argv)
	#创建一个窗口
	mainWindow = QMainWindow()
	#向主窗口上添加控件
	ui = text.Ui_MainWindow()
	ui.setupUi(mainWindow)
	mainWindow.show()
	
	#进入程序的主循环,并通过exit函数确保主循环安全结束
	sys.exit(app.exec_())
\end{Python}

需要在Qt Designer中设置信号(signal)与槽(slot),这是Qt的核心,也是PyQt的核心机制

信号:是由对象或控件发射出去的消息。当单击按钮时,按钮就会向外部发送单击的消息,这些发送出去的信号需要一些代码进行拦截,这些代码就是槽。槽本质是一个函数或方法。

信号可以理解为事件,槽可以理解为事件函数,需要将信号与槽绑定。一个信号可以和多个槽绑定,一个槽可以拦截多个信号。

主窗口有3种类型:

QMainWindow,可以包含菜单栏,工具栏,状态栏和标题栏,是最常见的窗口形式

QDialog:是对话窗口的基类,没有菜单栏,工具栏,状态栏。

QWidget:不确定窗口的用途,就使用QWidget。
\begin{Python}{代码打的第一个例子}
import sys
from PyQt5.QtWidgets import QMainWindow,QApplication
from PyQt5.QtGui import QIcon

class FirstMainWin(QMainWindow):
	def __init__(self):
		super(FirstMainWin,self).__init__()
		
		#设置主窗口的标题
		self.setWindowTitle('第一个主窗口应用')
		#设置窗口的尺寸
		self.resize(400,300)
		self.status = self.statusBar()
		self.status.showMessage('只存在5秒的消息',5000)

if __name__=='__main__':
app = QApplication(sys.argv)
app.setWindowIcon(QIcon ('517.ico'))
main = FirstMainWin()
main.show()

sys.exit(app.exec_())
	
\end{Python}
\end{document}