\documentclass[11pt]{article}
\usepackage{ctex}

\usepackage[left=1.25in,right=1.25in,top=1in,bottom=1in]{geometry}

\input{mp-style2.tex}

\usepackage{fontspec}  % for Consolas

\begin{document}
\section{scipy.spatial.transform}
\subsection{Rotation}
这个类提供了一个接口来初始化和表示旋转
\subsubsection{as\_rotvec}
旋转向量是与旋转轴同向的 3 维向量,其范数给出旋转角度
\begin{Python}{给定旋转角}
from scipy.stats import special_ortho_group
from scipy.spatial.transform import Rotation

rand_rot = special_ortho_group.rvs(3)       #随机的旋转矩阵
axis_rot= Rotation.from_matrix(rand_rot)    #初始化生成的旋转矩阵
axis_angle = Rotation.as_rotvec(axis_rot)   #旋转角度

print(axis_angle)

#OUTPUT:
#       [-0.01449134  0.30742885  1.34225116]
\end{Python}
\subsubsection{as\_matrix}
可以使用旋转矩阵表示3D旋转,旋转矩阵是3x3实正交矩阵,行列式等于+1
\begin{Python}{整体例子}
from scipy.spatial.transform import Rotation
from scipy.stats import special_ortho_group

rand_rot = special_ortho_group.rvs(3)
axis_angle = Rotation.as_rotvec(Rotation.from_matrix(rand_rot))
print(axis_angle)
rot_mag = 45
axis_angle *= rot_mag / 180.0
print(axis_angle)
rand_rot = Rotation.from_rotvec(axis_angle)
rand_rot1 = rand_rot.as_matrix()	#表示单次旋转
print(rand_rot1)

#OUTPUT:
#       [-0.94779163  2.85219495  0.34512507]
#
#       [-0.23694791  0.71304874  0.08628127]
#
#       [[ 0.75412225 -0.15881423  0.63724224]
#        [-0.00223926  0.96969261  0.24431786]
#        [-0.65673025 -0.18567249  0.73091115]]
\end{Python}
\subsubsection{from\_matrix}
3维旋转矩阵可以用3 x 3适当的正交矩阵表示。

如果输入不是适当的正交,则使用中描述的方法创建近似值。
\begin{Python}{生成正交矩阵}
from scipy.stats import special_ortho_group
from scipy.spatial.transform import Rotation

rand_rot = special_ortho_group.rvs(3)       #随机的旋转矩阵
axis_rot = Rotation.from_matrix(rand_rot)   #初始化生成的旋转矩阵

print(rand_rot)
#OUTPUT:
#       [[ 0.32443209  0.58884804 -0.74027144]
#        [ 0.79035836 -0.59872167 -0.12986922]
#        [-0.51968979 -0.54294598 -0.6596455 ]]

print(axis_rot)
#OUTPUT:
#       <scipy.spatial.transform.rotation.Rotation object at 0x000001DD6C65BD50>
\end{Python}
\subsubsection{from\_rotvec}
从旋转向量初始化。

旋转向量是与旋转轴同向的 3 维向量,其范数给出旋转角度.

import numpy as np
from scipy.spatial.transform import Rotation as R

\begin{Python}{给定旋转角}
import numpy as np
from scipy.spatial.transform import Rotation as R

r = R.from_rotvec(45 * np.array([0, 1, 0]))
b = r.as_rotvec()							#配套使用的

print(b)

#OUTPUT:
#       [0.         1.01770285 0.        ]
\end{Python}
\subsubsection{from\_quat}
从四元数初始化。

可以使用单位范数四元数表示 3D 旋转

每一行都是一个(可能是非单位范数)标量最后 (x, y, z, w) 格式的四元数。

每个四元数都将归一化为单位范数。

\begin{Python}{四元数组单位化}
from scipy.spatial.transform import Rotation as R

r = R.from_quat([0, 0, 2, 1])
a = r.as_quat()
print(a)

#OUTPUT:
#       [0.         0.         0.89442719 0.4472136 ]
\end{Python}
\section{scipy.stats}
\subsection{special\_ortho\_group}
返回从haar分布(SO(n) 上唯一的均匀分布)绘制的随机旋转矩阵
\begin{Python}{绘制随机旋转矩阵}
from scipy.stats import special_ortho_group

x = special_ortho_group.rvs(3)		#指定矩阵大小,随机生成矩阵
print(x)

#OUTPUT:
#		[[-0.48455797  0.03507638  0.87405562]
#		 [-0.2471805   0.95298494 -0.17527551]
#		 [-0.83910987 -0.30098065 -0.45310625]]
\end{Python}



\end{document}