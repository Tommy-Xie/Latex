\documentclass[11pt]{article}
\usepackage{ctex}

\usepackage[left=1.25in,right=1.25in,top=1in,bottom=1in]{geometry}

% Copyright 20120 Liutao Tian, MIT License
% https://github.com/andy123t/code-latex-style/

\usepackage{listings,color}

% Matlab highlight color settings
%\definecolor{mBasic}{RGB}{248,248,242}       % default
\definecolor{mKeyword}{RGB}{0,0,255}          % bule
\definecolor{mString}{RGB}{160,32,240}        % purple
\definecolor{mComment}{RGB}{34,139,34}        % green
\definecolor{mBackground}{RGB}{245,245,245}   % lightgrey
\definecolor{mNumber}{RGB}{134,145,148}       % gray

\definecolor{Numberbg}{RGB}{237,240,241}     % lightgrey

% Python highlight color settings
%\definecolor{pBasic}{RGB}{248, 248, 242}     % default
\definecolor{pKeyword}{RGB}{228,0,128}        % magenta
\definecolor{pString}{RGB}{148,0,209}         % purple
\definecolor{pComment}{RGB}{117,113,94}       % gray
\definecolor{pIdentifier}{RGB}{166, 226, 46}  %
\definecolor{pBackground}{RGB}{245,245,245}   % lightgrey
\definecolor{pNumber}{RGB}{134,145,148}       % gray

\lstnewenvironment{Python}[1]{
	\lstset{language=python,               % choose the language of the code
		xleftmargin=30pt,
		xrightmargin=10pt,
		frame=l,
		framesep=15pt,%framerule=0pt,  % sets the frame style
		%frame=shadowbox,rulesepcolor=\color{red!20!green!20!blue!20},
		%basicstyle=\small\ttfamily,          % sets font style for the code
		basicstyle=\footnotesize\fontspec{Consolas},
		keywordstyle=\color{pKeyword},       % sets color for keywords
		stringstyle=\color{pString},         % sets color for strings
		commentstyle=\color{pComment},       % sets color for comments
		backgroundcolor=\color{pBackground}, % choose the background color
		title=#1,                            %\lstname show the filename of files
		emph={format_string,eff_ana_bf,permute,eff_ana_btr},
		emphstyle=\color{pIdentifier}
		showspaces=false,                    % show spaces adding particular underscores
		showstringspaces=false,              % underline spaces within strings
		showtabs=false,                      % show tabs within strings adding particular underscores
		tabsize=4,                           % sets default tabsize to 2 spaces
		captionpos=t,                        % sets the caption-position to bottom
		breaklines=true,                     % sets automatic line breaking
		framexleftmargin=5pt,
		fillcolor=\color{Numberbg},
		rulecolor=\color{Numberbg},
		numberstyle=\tiny\color{pNumber},
		numbersep=9pt,                      % how far the line-numbers are from the code
		numbers=left,                        % where to put the line-numbers
		stepnumber=1,                        % the step between two line-numbers.
}}{}

\lstnewenvironment{Python1}[1]{
\lstset{language=python,               % choose the language of the code
  xleftmargin=30pt,
  xrightmargin=10pt,
  frame=l,
  framesep=15pt,%framerule=0pt,  % sets the frame style
  %frame=shadowbox,rulesepcolor=\color{red!20!green!20!blue!20},
  %basicstyle=\small\ttfamily,          % sets font style for the code
  basicstyle=\footnotesize\fontspec{Consolas},
  keywordstyle=\color{pKeyword},       % sets color for keywords
  stringstyle=\color{pString},         % sets color for strings
  commentstyle=\color{pComment},       % sets color for comments
  backgroundcolor=\color{pBackground}, % choose the background color
  title=#1,                            %\lstname show the filename of files
  emph={format_string,eff_ana_bf,permute,eff_ana_btr},
  emphstyle=\color{pIdentifier}
  showspaces=false,                    % show spaces adding particular underscores
  showstringspaces=false,              % underline spaces within strings
  showtabs=false,                      % show tabs within strings adding particular underscores
  tabsize=4,                           % sets default tabsize to 2 spaces
  captionpos=t,                        % sets the caption-position to bottom
  breaklines=true,                     % sets automatic line breaking
  framexleftmargin=5pt,
  fillcolor=\color{Numberbg},
  rulecolor=\color{Numberbg},
  numberstyle=\tiny\color{pNumber},
  numbersep=9pt,                      % how far the line-numbers are from the code
  numbers=left,                        % where to put the line-numbers
  stepnumber=1,                        % the step between two line-numbers.
}}{}




\usepackage{fontspec}  % for Consolas

\begin{document}
\section{scipy.spatial.transform}
\subsection{Rotation}
这个类提供了一个接口来初始化和表示旋转
\subsubsection{as\_rotvec}
旋转向量是与旋转轴同向的 3 维向量,其范数给出旋转角度
\begin{Python}{给定旋转角}
from scipy.stats import special_ortho_group
from scipy.spatial.transform import Rotation

rand_rot = special_ortho_group.rvs(3)       #随机的旋转矩阵
axis_rot= Rotation.from_matrix(rand_rot)    #初始化生成的旋转矩阵
axis_angle = Rotation.as_rotvec(axis_rot)   #旋转角度

print(axis_angle)

#OUTPUT:
#       [-0.01449134  0.30742885  1.34225116]
\end{Python}
\subsubsection{as\_matrix}
可以使用旋转矩阵表示3D旋转,旋转矩阵是3x3实正交矩阵,行列式等于+1
\begin{Python}{整体例子}
from scipy.spatial.transform import Rotation
from scipy.stats import special_ortho_group

rand_rot = special_ortho_group.rvs(3)
axis_angle = Rotation.as_rotvec(Rotation.from_matrix(rand_rot))
print(axis_angle)
rot_mag = 45
axis_angle *= rot_mag / 180.0
print(axis_angle)
rand_rot = Rotation.from_rotvec(axis_angle)
rand_rot1 = rand_rot.as_matrix()	#表示单次旋转
print(rand_rot1)

#OUTPUT:
#       [-0.94779163  2.85219495  0.34512507]
#
#       [-0.23694791  0.71304874  0.08628127]
#
#       [[ 0.75412225 -0.15881423  0.63724224]
#        [-0.00223926  0.96969261  0.24431786]
#        [-0.65673025 -0.18567249  0.73091115]]
\end{Python}
\subsubsection{from\_matrix}
3维旋转矩阵可以用3 x 3适当的正交矩阵表示。

如果输入不是适当的正交,则使用中描述的方法创建近似值。
\begin{Python}{生成正交矩阵}
from scipy.stats import special_ortho_group
from scipy.spatial.transform import Rotation

rand_rot = special_ortho_group.rvs(3)       #随机的旋转矩阵
axis_rot = Rotation.from_matrix(rand_rot)   #初始化生成的旋转矩阵

print(rand_rot)
#OUTPUT:
#       [[ 0.32443209  0.58884804 -0.74027144]
#        [ 0.79035836 -0.59872167 -0.12986922]
#        [-0.51968979 -0.54294598 -0.6596455 ]]

print(axis_rot)
#OUTPUT:
#       <scipy.spatial.transform.rotation.Rotation object at 0x000001DD6C65BD50>
\end{Python}
\subsubsection{from\_rotvec}
从旋转向量初始化。

旋转向量是与旋转轴同向的 3 维向量,其范数给出旋转角度.

import numpy as np
from scipy.spatial.transform import Rotation as R

\begin{Python}{给定旋转角}
import numpy as np
from scipy.spatial.transform import Rotation as R

r = R.from_rotvec(45 * np.array([0, 1, 0]))
b = r.as_rotvec()							#配套使用的

print(b)

#OUTPUT:
#       [0.         1.01770285 0.        ]
\end{Python}
\subsubsection{from\_quat}
从四元数初始化。

可以使用单位范数四元数表示 3D 旋转

每一行都是一个(可能是非单位范数)标量最后 (x, y, z, w) 格式的四元数。

每个四元数都将归一化为单位范数。

\begin{Python}{四元数组单位化}
from scipy.spatial.transform import Rotation as R

r = R.from_quat([0, 0, 2, 1])
a = r.as_quat()
print(a)

#OUTPUT:
#       [0.         0.         0.89442719 0.4472136 ]
\end{Python}
\section{scipy.stats}
\subsection{special\_ortho\_group}
返回从haar分布(SO(n) 上唯一的均匀分布)绘制的随机旋转矩阵
\begin{Python}{绘制随机旋转矩阵}
from scipy.stats import special_ortho_group

x = special_ortho_group.rvs(3)		#指定矩阵大小,随机生成矩阵
print(x)

#OUTPUT:
#		[[-0.48455797  0.03507638  0.87405562]
#		 [-0.2471805   0.95298494 -0.17527551]
#		 [-0.83910987 -0.30098065 -0.45310625]]
\end{Python}



\end{document}