\documentclass[11pt]{article}
\usepackage{ctex}

\usepackage[left=1.25in,right=1.25in,top=1in,bottom=1in]{geometry}

\input{mp-style2.tex}

\usepackage{fontspec}  % for Consolas

\begin{document}
\section{初始}
yacs库,用于为一个系统构建配置文件
	\begin{Python}{安装}
	pip install yacs
	\end{Python}
	\begin{Python}{导入}
	from yacs.config import CfgNode as CN
	\end{Python}
创建配置节点。需要创造CN()这个作为容器来装载我们的参数,这个容器可以嵌套
	\begin{Python}{create config node}
	from yacs.config import CfgNode as CN

	__C = CN()
	__C.name = 'test'
	__C.model = CN() #嵌套使用
	__C.model.backbone = 'resnet'
	__C.model.depth = 18
	
	print(__C)
	
	#OUTPUT:
	#		name: test
	#		model:
	#			backbone: resnet
	#			depth: 18
	\end{Python}
\section{内置函数}
使用上文中的$\_\_C$作为已经创建的配置文件。
\subsection{clone()}
返回一个复制配置文件,因此默认值不会被更改。
	\begin{Python}{克隆}
	def get_cfg_defaults():
		return __C.clone()
	\end{Python}
\subsection{clear()}
清空你的配置文件,你将得到None作为结果。
	\begin{Python}{清除}
	print(__C.clear())
		
	#OUTPUT:
	#		None
	\end{Python}
\subsection{merge from file()}
对于不同的实验,你有不同的超参设置,所以你可以使用yaml文件来管理不同的配置文件,然后使用merge\_from\_file()这个方法,这个会比较每个实验特有的配置信息和默认参数的区别,会将默认参数与特定参数不同的部分,用特定参数覆盖。
	\begin{Python}{加载不同的配置文件}
	__C.merge_from_file("./test_config.yaml")
	\end{Python}
注意点:

你需要合并的yaml文件中,不能有default参数中不存在的参数,不然会报错,但是可以比default中设定的参数少,比如default文件中有name参数,这是不需要特定改动的,你可以在yaml中不设置name这个key。
	\begin{Python}{补充}
	from yacs.config import CfgNode as CN
	# default cfgs
	__C = CN()
	__C.name = 'test'
	__C.model = CN()
	__C.model.backbone = 'resnet'
	__C.model.depth = 18
	
	# yaml cfgs
	# 不报错的情况1:参数和default中一样多,并且层级关系一致
	name: test
	model:
	backbone: resnet
	depth: 18
	
	# 不报错的情况2:参数可以比default中少,以下例子就不包含name和model.backbone
	model: 
	depth: 34
	
	# 报错的情况1:以下多了model.batch_normalization这个额外的key,这在default中是不存在的
	name: test
	model:
	backbone: resnet
	depth: 29
	batch_normalization: True
	
	# 报错的情况2:关键词不一致,这里的关键词是na_me,而default中是name
	na_me: test
	\end{Python}
\subsection{merge from list()}
可以用list来传递参数
	\begin{Python}{列表传递参数}
	from yacs.config import CfgNode as CN
	__C = CN()
	__C.name = 'test'
	__C.model = CN()
	__C.model.backbone = 'resnet'
	__C.model.depth = 18
	print(__C)
	
	#OUTPUT:
	#		model:
	#		backbone: resnet
	#		depth: 18
	#		name: test
	
	opts = ["name", 'test_name', "model.backbone", "vgg"]
	__C.merge_from_list(opts)
	print(__C)
	
	#OUTPUT:
	#	model:
	#	backbone: vgg
	#	depth: 18
	#	name: test_name
	\end{Python}
\subsection{freeze()}
冻结配置后,不能修改配置。
	\begin{Python}{冻结参数}
		from yacs.config import CfgNode as CN
		
		__C = CN()
		__C.name = 'test'
		__C.model = CN()
		__C.model.backbone = 'resnet'
		__C.model.depth = 18
		
		# freeze the config
		__C.freeze()
		# try to change the name's value, raise an error
		__C.name = 'test2'  # error
	\end{Python}
\subsection{defrost()}
解冻配置,可以修改配置。
\begin{Python}{解冻参数}
	from yacs.config import CfgNode as CN
	
	__C = CN()
	__C.name = 'test'
	__C.model = CN()
	__C.model.backbone = 'resnet'
	__C.model.depth = 18
	
	# freeze the config
	__C.freeze()
	# try to change the name's value, raise an error
	__C.name = 'test2'  # error
	
	__C.defrost() # not freeze cfgs, after this operation you can change the value
	__C.name = 'test2'  # work
\end{Python}
\end{document}