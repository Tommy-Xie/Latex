\documentclass[11pt]{article}
\usepackage{ctex}

\usepackage[left=1.25in,right=1.25in,top=1in,bottom=1in]{geometry}

% Copyright 20120 Liutao Tian, MIT License
% https://github.com/andy123t/code-latex-style/

\usepackage{listings,color}

% Matlab highlight color settings
%\definecolor{mBasic}{RGB}{248,248,242}       % default
\definecolor{mKeyword}{RGB}{0,0,255}          % bule
\definecolor{mString}{RGB}{160,32,240}        % purple
\definecolor{mComment}{RGB}{34,139,34}        % green
\definecolor{mBackground}{RGB}{245,245,245}   % lightgrey
\definecolor{mNumber}{RGB}{134,145,148}       % gray

\definecolor{Numberbg}{RGB}{237,240,241}     % lightgrey

% Python highlight color settings
%\definecolor{pBasic}{RGB}{248, 248, 242}     % default
\definecolor{pKeyword}{RGB}{228,0,128}        % magenta
\definecolor{pString}{RGB}{148,0,209}         % purple
\definecolor{pComment}{RGB}{117,113,94}       % gray
\definecolor{pIdentifier}{RGB}{166, 226, 46}  %
\definecolor{pBackground}{RGB}{245,245,245}   % lightgrey
\definecolor{pNumber}{RGB}{134,145,148}       % gray

\lstnewenvironment{Python}[1]{
	\lstset{language=python,               % choose the language of the code
		xleftmargin=30pt,
		xrightmargin=10pt,
		frame=l,
		framesep=15pt,%framerule=0pt,  % sets the frame style
		%frame=shadowbox,rulesepcolor=\color{red!20!green!20!blue!20},
		%basicstyle=\small\ttfamily,          % sets font style for the code
		basicstyle=\footnotesize\fontspec{Consolas},
		keywordstyle=\color{pKeyword},       % sets color for keywords
		stringstyle=\color{pString},         % sets color for strings
		commentstyle=\color{pComment},       % sets color for comments
		backgroundcolor=\color{pBackground}, % choose the background color
		title=#1,                            %\lstname show the filename of files
		emph={format_string,eff_ana_bf,permute,eff_ana_btr},
		emphstyle=\color{pIdentifier}
		showspaces=false,                    % show spaces adding particular underscores
		showstringspaces=false,              % underline spaces within strings
		showtabs=false,                      % show tabs within strings adding particular underscores
		tabsize=4,                           % sets default tabsize to 2 spaces
		captionpos=t,                        % sets the caption-position to bottom
		breaklines=true,                     % sets automatic line breaking
		framexleftmargin=5pt,
		fillcolor=\color{Numberbg},
		rulecolor=\color{Numberbg},
		numberstyle=\tiny\color{pNumber},
		numbersep=9pt,                      % how far the line-numbers are from the code
		numbers=left,                        % where to put the line-numbers
		stepnumber=1,                        % the step between two line-numbers.
}}{}

\lstnewenvironment{Python1}[1]{
\lstset{language=python,               % choose the language of the code
  xleftmargin=30pt,
  xrightmargin=10pt,
  frame=l,
  framesep=15pt,%framerule=0pt,  % sets the frame style
  %frame=shadowbox,rulesepcolor=\color{red!20!green!20!blue!20},
  %basicstyle=\small\ttfamily,          % sets font style for the code
  basicstyle=\footnotesize\fontspec{Consolas},
  keywordstyle=\color{pKeyword},       % sets color for keywords
  stringstyle=\color{pString},         % sets color for strings
  commentstyle=\color{pComment},       % sets color for comments
  backgroundcolor=\color{pBackground}, % choose the background color
  title=#1,                            %\lstname show the filename of files
  emph={format_string,eff_ana_bf,permute,eff_ana_btr},
  emphstyle=\color{pIdentifier}
  showspaces=false,                    % show spaces adding particular underscores
  showstringspaces=false,              % underline spaces within strings
  showtabs=false,                      % show tabs within strings adding particular underscores
  tabsize=4,                           % sets default tabsize to 2 spaces
  captionpos=t,                        % sets the caption-position to bottom
  breaklines=true,                     % sets automatic line breaking
  framexleftmargin=5pt,
  fillcolor=\color{Numberbg},
  rulecolor=\color{Numberbg},
  numberstyle=\tiny\color{pNumber},
  numbersep=9pt,                      % how far the line-numbers are from the code
  numbers=left,                        % where to put the line-numbers
  stepnumber=1,                        % the step between two line-numbers.
}}{}




\usepackage{fontspec}  % for Consolas

\begin{document}
\section{初始}
yacs库,用于为一个系统构建配置文件
	\begin{Python}{安装}
	pip install yacs
	\end{Python}
	\begin{Python}{导入}
	from yacs.config import CfgNode as CN
	\end{Python}
创建配置节点。需要创造CN()这个作为容器来装载我们的参数,这个容器可以嵌套
	\begin{Python}{create config node}
	from yacs.config import CfgNode as CN

	__C = CN()
	__C.name = 'test'
	__C.model = CN() #嵌套使用
	__C.model.backbone = 'resnet'
	__C.model.depth = 18
	
	print(__C)
	
	#OUTPUT:
	#		name: test
	#		model:
	#			backbone: resnet
	#			depth: 18
	\end{Python}
\section{内置函数}
使用上文中的$\_\_C$作为已经创建的配置文件。
\subsection{clone()}
返回一个复制配置文件,因此默认值不会被更改。
	\begin{Python}{克隆}
	def get_cfg_defaults():
		return __C.clone()
	\end{Python}
\subsection{clear()}
清空你的配置文件,你将得到None作为结果。
	\begin{Python}{清除}
	print(__C.clear())
		
	#OUTPUT:
	#		None
	\end{Python}
\subsection{merge from file()}
对于不同的实验,你有不同的超参设置,所以你可以使用yaml文件来管理不同的配置文件,然后使用merge\_from\_file()这个方法,这个会比较每个实验特有的配置信息和默认参数的区别,会将默认参数与特定参数不同的部分,用特定参数覆盖。
	\begin{Python}{加载不同的配置文件}
	__C.merge_from_file("./test_config.yaml")
	\end{Python}
注意点:

你需要合并的yaml文件中,不能有default参数中不存在的参数,不然会报错,但是可以比default中设定的参数少,比如default文件中有name参数,这是不需要特定改动的,你可以在yaml中不设置name这个key。
	\begin{Python}{补充}
	from yacs.config import CfgNode as CN
	# default cfgs
	__C = CN()
	__C.name = 'test'
	__C.model = CN()
	__C.model.backbone = 'resnet'
	__C.model.depth = 18
	
	# yaml cfgs
	# 不报错的情况1:参数和default中一样多,并且层级关系一致
	name: test
	model:
	backbone: resnet
	depth: 18
	
	# 不报错的情况2:参数可以比default中少,以下例子就不包含name和model.backbone
	model: 
	depth: 34
	
	# 报错的情况1:以下多了model.batch_normalization这个额外的key,这在default中是不存在的
	name: test
	model:
	backbone: resnet
	depth: 29
	batch_normalization: True
	
	# 报错的情况2:关键词不一致,这里的关键词是na_me,而default中是name
	na_me: test
	\end{Python}
\subsection{merge from list()}
可以用list来传递参数
	\begin{Python}{列表传递参数}
	from yacs.config import CfgNode as CN
	__C = CN()
	__C.name = 'test'
	__C.model = CN()
	__C.model.backbone = 'resnet'
	__C.model.depth = 18
	print(__C)
	
	#OUTPUT:
	#		model:
	#		backbone: resnet
	#		depth: 18
	#		name: test
	
	opts = ["name", 'test_name', "model.backbone", "vgg"]
	__C.merge_from_list(opts)
	print(__C)
	
	#OUTPUT:
	#	model:
	#	backbone: vgg
	#	depth: 18
	#	name: test_name
	\end{Python}
\subsection{freeze()}
冻结配置后,不能修改配置。
	\begin{Python}{冻结参数}
		from yacs.config import CfgNode as CN
		
		__C = CN()
		__C.name = 'test'
		__C.model = CN()
		__C.model.backbone = 'resnet'
		__C.model.depth = 18
		
		# freeze the config
		__C.freeze()
		# try to change the name's value, raise an error
		__C.name = 'test2'  # error
	\end{Python}
\subsection{defrost()}
解冻配置,可以修改配置。
\begin{Python}{解冻参数}
	from yacs.config import CfgNode as CN
	
	__C = CN()
	__C.name = 'test'
	__C.model = CN()
	__C.model.backbone = 'resnet'
	__C.model.depth = 18
	
	# freeze the config
	__C.freeze()
	# try to change the name's value, raise an error
	__C.name = 'test2'  # error
	
	__C.defrost() # not freeze cfgs, after this operation you can change the value
	__C.name = 'test2'  # work
\end{Python}
\end{document}