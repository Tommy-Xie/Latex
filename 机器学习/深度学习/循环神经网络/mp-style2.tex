% Copyright 20120 Liutao Tian, MIT License
% https://github.com/andy123t/code-latex-style/

\usepackage{listings,color}

% Matlab highlight color settings
%\definecolor{mBasic}{RGB}{248,248,242}       % default
\definecolor{mKeyword}{RGB}{0,0,255}          % bule
\definecolor{mString}{RGB}{160,32,240}        % purple
\definecolor{mComment}{RGB}{34,139,34}        % green
\definecolor{mBackground}{RGB}{245,245,245}   % lightgrey
\definecolor{mNumber}{RGB}{134,145,148}       % gray

\definecolor{Numberbg}{RGB}{237,240,241}     % lightgrey

% Python highlight color settings
%\definecolor{pBasic}{RGB}{248, 248, 242}     % default
\definecolor{pKeyword}{RGB}{228,0,128}        % magenta
\definecolor{pString}{RGB}{148,0,209}         % purple
\definecolor{pComment}{RGB}{117,113,94}       % gray
\definecolor{pIdentifier}{RGB}{166, 226, 46}  %
\definecolor{pBackground}{RGB}{245,245,245}   % lightgrey
\definecolor{pNumber}{RGB}{134,145,148}       % gray

\lstnewenvironment{Matlab}[1]{
\lstset{language=Matlab,               % choose the language of the code
  %frame=tlbr,
  xleftmargin=30pt,
  xrightmargin=10pt,
  frame=l,
  framesep=15pt,%framerule=0pt,  % sets the frame style
  %frame=shadowbox,rulesepcolor=\color{red!20!green!20!blue!20},
  basicstyle=\small\ttfamily,
  keywordstyle={\color{mKeyword}},     % sets color for keywords
  stringstyle={\color{mString}},       % sets color for strings
  commentstyle={\color{mComment}},     % sets color for comments
  backgroundcolor=\color{mBackground}, % choose the background color
  title=#1,                            % \lstname show the filename of files
  keywords={break,case,catch,classdef,continue,else,elseif,end,for,
  function,global,if,otherwise,parfor,persistent,return,spmd,switch,try,while},
  showspaces=false,                    % show spaces adding particular underscores
  showstringspaces=false,              % underline spaces within strings
  showtabs=false,                      % show tabs within strings adding particular underscores
  tabsize=4,                           % sets default tabsize to 2 spaces
  captionpos=t,                        % sets the caption-position to bottom
  breaklines=true,                     % sets automatic line breaking
  framexleftmargin=5pt,
  fillcolor=\color{Numberbg},
  rulecolor=\color{Numberbg},
  numberstyle=\tiny\color{mNumber},
  numbersep=9pt,                      % how far the line-numbers are from the code
  numbers=left,                        % where to put the line-numbers
  stepnumber=1,                        % the step between two line-numbers.
}}{\lstname}

\lstnewenvironment{Python}[1]{
\lstset{language=python,               % choose the language of the code
  xleftmargin=30pt,
  xrightmargin=10pt,
  frame=l,
  framesep=15pt,%framerule=0pt,  % sets the frame style
  %frame=shadowbox,rulesepcolor=\color{red!20!green!20!blue!20},
  %basicstyle=\small\ttfamily,          % sets font style for the code
  basicstyle=\footnotesize\fontspec{Consolas},
  keywordstyle=\color{pKeyword},       % sets color for keywords
  stringstyle=\color{pString},         % sets color for strings
  commentstyle=\color{pComment},       % sets color for comments
  backgroundcolor=\color{pBackground}, % choose the background color
  title=#1,                            %\lstname show the filename of files
  emph={format_string,eff_ana_bf,permute,eff_ana_btr},
  emphstyle=\color{pIdentifier}
  showspaces=false,                    % show spaces adding particular underscores
  showstringspaces=false,              % underline spaces within strings
  showtabs=false,                      % show tabs within strings adding particular underscores
  tabsize=4,                           % sets default tabsize to 2 spaces
  captionpos=t,                        % sets the caption-position to bottom
  breaklines=true,                     % sets automatic line breaking
  framexleftmargin=5pt,
  fillcolor=\color{Numberbg},
  rulecolor=\color{Numberbg},
  numberstyle=\tiny\color{pNumber},
  numbersep=9pt,                      % how far the line-numbers are from the code
  numbers=left,                        % where to put the line-numbers
  stepnumber=1,                        % the step between two line-numbers.
}}{}


